\documentclass[american,firsttime]{dfgproposal}

% This template is based on the DFG RTF form (see Readme on GitHub for last accessed date, and version in the header of the compiled PDF)
%
% Author: Karl-Ludwig Besser, k.besser@tu-bs.de
% (based on a template by Martin Hoelzer, hoelzer.martin@gmail.com)

\usepackage{blindtext}

\usepackage{graphicx}

\usepackage{pgfgantt}


\addbibresource{literature.bib}
\renewcommand*{\bibfont}{\small}

\newcommand{\todo}[1]{\xspace{\textcolor{red}{[TODO: #1]}}\xspace}
\newcommand{\cp}[1]{\xspace{\textcolor{blue}{[COPIED: #1]}}\xspace}
\newcommand{\note}[1]{{\color{orange}{#1}}}

%%%%%%%%%%%%%%%%%%%%%%%%%%%%%%%%%%%%%%%%%%%%%%%%%%%%%%%%%%%%%%%%%%%%%%%%%%%%
%% Declare your Bibliography
%
%\nocite{}

% for references that should be bold, add their corresponding citation keys to \addtocategory{important}{%...}
\addtocategory{important}{%
Hoelzer:16,Hoelzer:17,Smith2023b,Smith2023c,Smith2023d, % don't miss the comma after the last entry
}

\usepackage[
	colorlinks=true,
	allcolors=.,
	linkcolor=MidnightBlue,
	citecolor=MidnightBlue,
	urlcolor=MidnightBlue,
	pdfauthor={},
	pdfpagelayout=SinglePage,
	bookmarks,
	bookmarksopen
]{hyperref}

\usepackage[numbered]{bookmark}


\title{Project Title}
\author[% short author name for short title on first page
	Prof. Name Author\\
	Institute 1, University 1 (Abbr)
	\and
	Author 2\\
	Institute 2, University 2 (Abbr2)
]{Prof. Name Author\\
	Institute 1\\University 1\\Street Number, 012345 Germany
	\and
	Author 2\\
	Institute 2\\University 2\\Street Number, 012345 Germany
}
\begin{document}
	\pagenumbering{arabic}
	\maketitle
	
	\note{Sections 1--3 must not exceed 17 pages in total.}
	
	%%%%%%%%%%%%%%%%%%%%%%%%%%%%%%%%%%%%%%%%%%%%%%%%%%%%%%%%%%%%%%%%%%%%%%%%%%%%%
	%%%%  STATE OF THE ART AND PRELIMINARY WORK %%%%%%%%%%%%%%%%%%%%%%%%%%%%%%%%%
	%%%%%%%%%%%%%%%%%%%%%%%%%%%%%%%%%%%%%%%%%%%%%%%%%%%%%%%%%%%%%%%%%%%%%%%%%%%%%
	\section{Starting Point}
%	\section{Ausgangslage}
	\label{sec:work-report}
	
	\subsection{State of the art and preliminary work}
%	\subsection{Stand der Forschung und eigene Vorarbeiten}
	This is a dummy citation~\cite{Hoelzer:17}. Yeah. And another
	one~\cite{Gerst:18}. Wuhuh.
	Since version 09/22 of the DFG template, there is just a single common
	bibliography in section \enquote{\ref{sec:bib}~Project- and subject-related
	list of publications} for your own publications~\cite{Hoelzer:16, Desiro:18}
	and those of others. The former subsubsections \emph{Articles published by
	outlets with scientific quality \dots} and \emph{Other publications, both
	peer-reviewed and non-peer-reviewed} are deprecated.
	
	\blindtext[1]
	
	
	%%%%%%%%%%%%%%%%%%%%%%%%%%%%%%%%%%%%%%%%%%%%%%%%%%%%%%%%%%%%%%%%%%%%%%%%%%%%%
	%%%%  OBJECTIVES AND WP %%%%%%%%%%%%%%%%%%%%%%%%%%%%%%%%%%%%%%%%%%%%%%%%%%%%%
	%%%%%%%%%%%%%%%%%%%%%%%%%%%%%%%%%%%%%%%%%%%%%%%%%%%%%%%%%%%%%%%%%%%%%%%%%%%%%
	\section{Objectives and work programme}
%	\section{Ziele und Arbeitsprogramm}
	
	\subsection{Anticipated total duration of the project}
%	\subsection{Voraussichtliche Gesamtdauer des Projekts}
	Financial support is requested for \todo{three years}.
	
	\subsection{Objectives}
%	\subsection{Ziele}
	\goal{Goal one is...}
	\goal{Goal two is...}
	
	\subsection{Work programme including proposed research methods}
%	\subsection{Arbeitsprogramm inkl. vorgesehener Untersuchungsmethoden}
	%For each applicant
	\todo{Describe the work packages... ca. 6--8 pages; ca. 4-8 WPs}
	
	%%%%%%%%%%%%%%%%%%%%%%%%%%%%%%%%%%%%%%%%%%%%%%%%%%%%%%%%%%%%%%%%%%%%%%%% 
	%%%%% BEGIN OF WORK PACKAGES %%%%%%%%%%%%%%%%%%%%%%%%%%%%%%%%%%%%%%%%%%%
	%%%%%%%%%%%%%%%%%%%%%%%%%%%%%%%%%%%%%%%%%%%%%%%%%%%%%%%%%%%%%%%%%%%%%%%% 
	\begin{workpackage}[Title of this WP]
		In this work package, we will do amazing things.
		
		In particular, we will...
		\blindtext
	\end{workpackage}
	\begin{wpsummary}
		\textbf{Aim:}
		In this work package, we do a lot of things...
		
		\textbf{Method:}
		Based on our unique 
		
		\textbf{Outcome:}
		\begin{itemize}
			\item Amazing
			\item Things
		\end{itemize}
	\end{wpsummary}
	
	\begin{workpackage}[Title of the Second WP]
		In this work package, we will do amazing things.
		
		In particular, we will...
		\blindtext
	\end{workpackage}
	\begin{wpsummary}
		\textbf{Aim:}
		In this work package, we do a lot of things...
		
		\textbf{Method:}
		Based on our unique 
		
		\textbf{Outcome:}
		\begin{itemize}
			\item Amazing
			\item Things
		\end{itemize}
	\end{wpsummary}
	
	%%%%%%%%%%%%%%%%%%%%%%%%%%%%%%%%%%%%%%%%%%%%%%%%%%%%%%%%%%%%%%%%%%%%%%%%%%%%%
	%%%%  END OF WORK PACKAGES  %%%%%%%%%%%%%%%%%%%%%%%%%%%%%%%%%%%%%%%%%%%%%%%%%
	%%%%%%%%%%%%%%%%%%%%%%%%%%%%%%%%%%%%%%%%%%%%%%%%%%%%%%%%%%%%%%%%%%%%%%%%%%%%%
	
	
	%%%%%%%%%%%%%%%%%%%%%%%%%%%%%%%%%%%%%%%%%%%%%%%%%%%%%%%%%%%%%%%%%%%%%%%%%%%%%
	%%%%  TIMELINE  %%%%%%%%%%%%%%%%%%%%%%%%%%%%%%%%%%%%%%%%%%%%%%%%%%%%%%%%%%%%%
	%%%%%%%%%%%%%%%%%%%%%%%%%%%%%%%%%%%%%%%%%%%%%%%%%%%%%%%%%%%%%%%%%%%%%%%%%%%%%
	%% This is not directly part of the DFG RTF template, but always helpful in my eyes
	\paragraph*{Time considerations}
	\todo{put timeline here, if needed}
	
	\input{ganttchart.tex}
	
	
	\subsection{Handling of research data}
%	\subsection{Umgang mit Forschungsdaten}
	Data generated during this project will be used for scientific publications in
	preprint and peer-reviewed journals. Therefore, all necessary data (such as raw
	sequencing data) will be deposited in publicly available repositories
	(\href{https://www.ncbi.nlm.nih.gov/geo/}{NCBI GEO} and
	\href{https://www.ncbi.nlm.nih.gov/sra}{SRA}, \href{https://osf.io/}{OSF}). To
	allow complete reproducibility of all analyses, the source code and executed
	comands will be also made publicly available
	(\href{https://github.com/}{GitHub}).  Thus, all relevant data will become
	accessible for future use. 
	\todo{FAIR principals}
	
	
	\subsection{Relevance of sex, gender and/or diversity}
%	\subsection{Relevanz von Geschlecht und/oder Vielfältigkeit}
	\todo{Text}
	
	\section{Project- and subject-related list of publications}
%	\section{Projekt- und themenbezogenes Literaturverzeichnis}
	\label{sec:bib}
	\note{Publications cited in sections 1 and 2, both your own publications and
	those of third parties. Please include DOIs if available or other persistant
	identifiers, preferably by stating the number or otherwise by URL. A maximum
	of ten of your own publications that are most relevant to the project may be
	highlighted. The font size should be at least 9\,pt.}
	\newrefcontext[sorting=none]
	\printbibliography[heading=none, env=bibliographyNUM]
	
	
	\clearpage
	\pagenumbering{Roman}
	
	%%%%%%%%%%%%%%%%%%%%%%%%%%%%%%%%%%%%%%%%%%%%%%%%%%%%%%%%%%%%%%%%%%%%%%%%%%%%%%%
	%%%%  SUPPLEMENT  (new since 04/2020) %%%%%%%%%%%%%%%%%%%%%%%%%%%%%%%%%%%%%%%%%
	%%%%%%%%%%%%%%%%%%%%%%%%%%%%%%%%%%%%%%%%%%%%%%%%%%%%%%%%%%%%%%%%%%%%%%%%%%%%%%%
	\note{Section 4 et seq. must not exceed 8 pages.}
	
	\section{Supplementary information on the research context}
%	\section{Begleitinformationen zum Forschungskontext}
	
	\subsection{Ethical and/or legal aspects of the project}
%	\subsection{Angaben zu ethischen und/oder rechtlichen Aspekten des Vorhabens}
	
	\subsubsection{General ethical aspects}
%	\subsubsection{Allgemeine ethische Aspekte}
	\todo{Text}
	
	\subsubsection{Descriptions of proposed investigations on humans, human materials or identifiable data}
%	\subsubsection{Erläuterungen zu den vorgesehenen Untersuchungen am Menschen, an vom Menschen entnommenem Material oder mit identifizierbaren Daten}
	\todo{Text}
	
	\subsubsection{Descriptions of proposed investigations involving experiments on animals}
%	\subsubsection{Erläuterungen zu den vorgesehenen Untersuchungen bei Versuchen an Tieren}
	\todo{Text}
	
	\subsubsection{Descriptions of projects involving genetic resources (or associated traditional knowledge) from a foreign country}
%	\subsubsection{Erläuterungen zu Forschungsvorhaben an genetischen Ressourcen (oder darauf bezogenem traditionellem Wissen) aus dem Ausland}
	\todo{Text}
	
	\subsubsection{Explanations regarding any possible safety-related aspects (\enquote{Dual Use Research of Concern}; foreign trade law)}
%	\subsubsection{Erläuterungen zu möglichen sicherheitsrelevanten Aspekten (\enquote{Dual-Use Research of Concern}; Außenwirtschaftsrecht)}
	\todo{Text}
	
	
	\subsection{Employment status information}
%	\subsection{Angaben zur Dienststellung}
	%% For each applicant, state the last name, first name, and employment status (including duration of contract and funding body, if on a fixed-term contract).
	\todo{Text}
	
	\subsection{First-time proposal data}
%	\subsection{Angaben zur Erstantragstellung}
	%% Only if applicable: Last name, first name of first-time applicant
	\todo{Text}
	
	
	
	%%%%%%%%%%%%%%%%%%%%%%%%%%%%%%%%%%%%%%%%%%%%%%%%%%%%%%%%%%%%%%%%%%%%%%%%%%%%%%%
	%%%%  PEOPLE/COLLABORATION  %%%%%%%%%%%%%%%%%%%%%%%%%%%%%%%%%%%%%%%%%%%%%%%%%%%
	%%%%%%%%%%%%%%%%%%%%%%%%%%%%%%%%%%%%%%%%%%%%%%%%%%%%%%%%%%%%%%%%%%%%%%%%%%%%%%%
	\subsection{Composition of the project group}
	%\subsection{Zusammensetzung der Projektarbeitsgruppe}
	%% List only those individuals who will work on the project but will not be paid out of the project funds. State each person’s name, academic title, employment status, and type of funding.
	The following individuals will work on the project but will not be paid out of
	the project funds:
	\begin{itemize}
		\item \todo{Person 1}
		\item \todo{Person 2}
	\end{itemize}
	
	
	\subsection{Researchers in Germany with whom you have agreed to cooperate on this project}
%	\subsection{Zusammenarbeit mit Wissenschaftlerinnen und Wissenschaftlern in Deutschland in diesem Projekt}
	\todo{Text}
	
	\subsection{Researchers abroad with whom you have agreed to cooperate on this project}
%	\subsection{Zusammenarbeit mit Wissenschaftlerinnen und Wissenschaftlern im Ausland in diesem Projekt}
	\todo{Text}
	
	\subsection{Researchers with whom you have collaborated scientifically within the past three years}
%	\subsection{Wissenschaftlerinnen und Wissenschaftler, mit denen in den letzten drei Jahren wissenschaftlich zusammengearbeitet wurde}
	% This information will help avoid potential conflicts of interest.
	\begin{itemize}
		\item \todo{Prof.~Dr.~Foo Bar, University XYZ, Germany}
	\end{itemize}
	
	\subsection{Project-relevant cooperation with commercial enterprises}
%	\subsection{Projektrelevante Zusammenarbeit mit erwerbswirtschaftlichen Unternehmen}
	%If applicable, please note the EU guidelines on state aid or contract your research institution in this regard.
	---\,None\,---
	
	\subsection{Project-relevant participation in commercial enterprises}
%	\subsection{Projektrelevante Beteiligungen an erwerbswirtschaftlichen Unternehmen}
	%Information on connections between the project and the production branch of the enterprise
	---\,None\,---
	
	\subsection{Scientific equipment}
%	\subsection{Apparative Ausstattung}
	%List larger instruments that will be available to you for the project. These may include large computer facilities if computing capacity will be needed.
	\todo{Text}
	
	\subsection{Other submissions}
%	\subsection{Weitere Antragstellungen}
	%List any funding proposals for this project and/or major instrumentation previously submitted to a third party.
	\todo{Text}
	
	
	
	\subsection{Other information}
%	\subsection{Weitere Angaben}
	%Please use this section for any additional information you feel is relevant which has not been provided elsewhere.
	In submitting a proposal to the DFG, I agree to adhere to the DFG's rules of good scientific practice and the \href{https://www.nature.com/articles/sdata201618}{FAIR principles}.
	
	
	%%%%%%%%%%%%%%%%%%%%%%%%%%%%%%%%%%%%%%%%%%%%%%%%%%%%%%%%%%%%%%%%%%%%%%%%%%%%%%%
	%%%%  REQUESTED MODULES/FUNDS  %%%%%%%%%%%%%%%%%%%%%%%%%%%%%%%%%%%%%%%%%%%%%%%%
	%%%%%%%%%%%%%%%%%%%%%%%%%%%%%%%%%%%%%%%%%%%%%%%%%%%%%%%%%%%%%%%%%%%%%%%%%%%%%%%
	\section{Requested modules/funds}
%	\section{Beantragte Module/Mittel}
	% Explain each item for each applicant (stating last name, first name).
	
	\subsection{Basic Module}
%	\subsection{Basismodul}
	
	\subsubsection{Funding for Staff}
%	\subsubsection{Personalmittel}
	\begin{funds}[funding for staff]
		The following staff positions are requested for \todo{xx} years each:
		
		\positionmul{Research associate, TV-L 13, 36 months}{5375}{36}
		\positionmul{Hiwi, TV-L 13, 12 months}{450}{12}
	\end{funds}
	
	
	\subsubsection{Direct Project Costs}
%	\subsubsection{Sachmittel}
	\begin{funds}[direct project costs]
		
		\subsubsubsection{Equipment up to 10\,000\,\euro, Software and Consumables}
%		\subsubsubsection{Geräte bis 10.000 Euro, Software und Verbrauchsmaterial}
		
		\position{Next-Generation Sequencing, Illumina HiSeq\,2500}{3000}
		
		\subsubsubsection{Travel Expenses}
%		\subsubsubsection{Reisemittel}
		We apply for a total of 10\,000\,\euro\ for travel expenses.
		\position{Travel}{10000}
		
		\subsubsubsection{Visiting Researchers \textnormal{(excluding Mercator Fellows)}}
%		\subsubsubsection{Mittel für wissenschaftliche Gäste \textnormal{(ausgenommen Mercator-Fellow)}}
		---\,None\,---
		
		\subsubsubsection{Expenses for Laboratory Animals}
%		\subsubsubsection{Mittel für Versuchstiere}
		---\,None\,---
		
		\subsubsubsection{Other Costs}
%		\subsubsubsection{Sonstige Mittel}
		---\,None\,---
		
		\subsubsubsection{Project-related publication expenses}
%		\subsubsubsection{Publikationsmittel}
		We apply for a total of 1500\,\euro\ for publication expenses (750\,\euro\ per
		year). Furthermore, we will submit articles to open access preprint repositories
		such as \href{https://www.biorxiv.org/}{\textit{bioRxiv}}.
		\position{Publication costs}{1500}
	\end{funds}
	
	\subsubsection{Instrumentation}
%	\subsubsection{Investitionsmittel}
	
	\subsubsubsection{Equipment exceeding 10\,000\,\euro}
%	\subsubsubsection{Geräte über 10.000 Euro}
	Our laboratory is well equipped; all necessary instruments are available.
	
	\subsubsubsection{Major Instrumentation exceeding 50\,000\,\euro}
%	\subsubsubsection{Großgeräte über 50.000 Euro}
	Our laboratory is well equipped; all necessary instruments are available.
	
	
	%%%%%%%%%%%%%%%%%%%%%%%%%%%%%%%%%%%%%%%%%%%%
	%% The following are additional/other modules one might want to apply for
	%% otherwise, just delete/comment
	%%%%%%%%%%%%%%%%%%%%%%%%%%%%%%%%%%%%%%%%%%%%
	
%	\vspace*{2cm}\todo{The following are additional/other modules one might want to apply for. Otherwise, just delete/comment}
	\subsection{Module Temporary Position for Principal Investigator}
%	\subsection{Modul Eigene Stelle}
	\todo{Text}
	
	\subsection{Module Replacement Funding}
%	\subsection{Modul Vertretung}
	\todo{Text}
	
	\subsection{Module Temporary Clinician Substitute}
%	\subsection{Modul Rotationsstellen}
	\todo{Text}
	
	\subsection{Module Mercator Fellows}
%	\subsection{Modul Mercator Fellow}
	\todo{Text}
	
	\subsection{Module Workshop Funding}
%	\subsection{Modul Projektspezifische Workshops}
	\todo{Text}
	
	\subsection{Module Public Relations}
%	\subsection{Modul Öffentlichkeitsarbeit}
	\todo{Text}
	
	\subsection{Module Standard Allowance for Gender Equality Measures}
%	\subsection{Modul Pauschale für Chancengleichheitsmaßnahmen}
	%{Please detail what measures are planned to promote diversity and equal opportunities.  If you are submitting your proposal for an individual research grant within a network, note that this standard allowance may only be applied for within the coordination project. The coordination project must combine all such requests in its calculation.}
	\todo{Text}
\end{document}
